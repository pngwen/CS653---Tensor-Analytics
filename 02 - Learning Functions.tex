% Uncomment for handout
\def\HANDOUT{}


\ifdefined\HANDOUT
\documentclass[handout]{beamer}
\usepackage{pgfpages}
\pgfpagesuselayout{4 on 1}[letterpaper,landscape,border shrink=5mm]
\else
\documentclass{beamer}
\fi

\mode<presentation>
{
  \usetheme{Warsaw}
  \definecolor{sered}{rgb}{0.78, 0.06, 0.18}
  \definecolor{richblack}{rgb}{0.0, 0.0, 0.0}
  \setbeamercolor{structure}{fg=sered,bg=richblack}
  %\setbeamercovered{transparent}
}


\usepackage[english]{babel}
\usepackage[latin1]{inputenc}
\usepackage{times}
\usepackage[T1]{fontenc}
\usepackage{tikz}
\usepackage{graphicx}
\usepackage[export]{adjustbox}
\usepackage{fancyvrb}
\usepackage{amsmath}
\usepackage{amssymb}
\usepackage{esvect}

\newcommand{\imagesource}[1]{{\centering\hfill\break\hbox{\scriptsize Image Source:\thinspace{\tiny\itshape #1}}\par}}
\newcommand{\image}[2]{%
        \begin{center}
        \includegraphics[max height = 0.55\textheight, max width = \textwidth]{images/#1}
        \linebreak
        {\tiny Image Source:\thinspace{\tiny #2}}
        \end{center}
}

\newenvironment{code}{%
 \VerbatimEnvironment
 \begin{adjustbox}{max width=\textwidth, max height=0.7\textheight}
 \begin{BVerbatim}
  }{
  \end{BVerbatim}
 \end{adjustbox}
}

\title{02 - Learning Functions}


\author{Robert Lowe}

\institute[Southeast Missouri State University] % (optional, but mostly needed)
{
  Department of Computer Science\\
  Southeast Missouri State University
}

\date[]{}
\subject{}

\pgfdeclareimage[height=1.0cm]{university-logo}{images/semo-logo}
\logo{\pgfuseimage{university-logo}}



\AtBeginSection[]
{
  \begin{frame}<beamer>{Outline}
    \tableofcontents[currentsection]
  \end{frame}
}


\begin{document}

\begin{frame}
  \titlepage
\end{frame}

\begin{frame}{Outline}
  \tableofcontents
\end{frame}


% Structuring a talk is a difficult task and the following structure
% may not be suitable. Here are some rules that apply for this
% solution: 

% - Exactly two or three sections (other than the summary).
% - At *most* three subsections per section.
% - Talk about 30s to 2min per frame. So there should be between about
%   15 and 30 frames, all told.

% - A conference audience is likely to know very little of what you
%   are going to talk about. So *simplify*!
% - In a 20min talk, getting the main ideas across is hard
%   enough. Leave out details, even if it means being less precise than
%   you think necessary.
% - If you omit details that are vital to the proof/implementation,
%   just say so once. Everybody will be happy with that.

\section{Feature Vectors}
\begin{frame}{Feature Vectors}
    \begin{columns}
    \column{0.5\textwidth}
        \begin{itemize}
            \item Suppose we want to learn how to respond to a sequence of $n$ variables.
            \item We can treat them as a vector.
            \item Each case in our set becomes a vector.
        \end{itemize}
    
    \column{0.5\textwidth}
        \[
        \vv{v} = \left(
        \begin{array}{c}
            v_1 \\
            v_2 \\
            \vdots\\
            v_n
        \end{array}
        \right)
        \]
    \end{columns}
\end{frame}

\begin{frame}{Objective Function}
    \begin{center}
    \begin{adjustbox}{height=0.1\textheight}
    \[
    f : \vv{v} \mapsto r
    \]
    \end{adjustbox}
    \end{center}
    \begin{description}
        \item[$f$] A function which maps feature vectors to responses
        \item[$\vv{v}$] The feature vector.
        \item[$r$] The response object. This could be a vector, label, scalar value, really any object.
    \end{description}
\end{frame}

\section{A Little Functional Analysis}

\begin{frame}{Learning a Function}
    \[
    f : \vv{v} \mapsto r
    \]
    \begin{itemize}
        \item For our purposes, ``functional analysis'' will be defined as learning the objective function for a set.
        \item Goals
        \begin{itemize}
            \item Compute Responses for Unknown Cases
            \item Identify Structures in Known Cases
        \end{itemize}
    \end{itemize}
\end{frame}

\begin{frame}{A Na\"ive Approach}
    \[
    \left[
    \begin{array}{cccc}
    v_{11} & v_{12} & \ldots & v_{1n} \\
    v_{21} & v_{22} & \ldots & v_{2n} \\
    \vdots & \vdots & \vdots & \vdots \\
    v_{m1} & v_{m2} & \ldots & v_{mn} \\
    \end{array}
    \right]
    \left[
    \begin{array}{c} 
        w_1 \\
        w_2 \\
        \vdots \\
        w_n
    \end{array}
    \right] =
    \left[
    \begin{array}{c} 
        r_1 \\
        r_2 \\
        \vdots \\
        r_m
    \end{array}
    \right]
    \]
    
    Attempt to combine variables as a weighted sum.
    \begin{enumerate}
        \item Treat $m$ case vectors as row vectors in an $m \times n$ matrix, $V$.
        \item Arrange weights as an $n$ element column vector, $W$.
        \item Arrange case responses as an $m$ element column vector, $R$.
        \item Solve for the weights.
        \begin{align*}
            VW &= R \\
            V^{-1}VW &= V^{-1}R \\
            W &= V^{-1}R
        \end{align*}
    \end{enumerate}
\end{frame}

\begin{frame}{The Trouble}
    \begin{itemize}
        \item We are making a big assumption! We are assuming that $\vv{r}$ is computable as:
        \[
        f(\vv{v}) = \sum_{i=1}^n w_i v_i
        \]
        \item For any linear system, one of the following must be true:
        \begin{enumerate}
            \item The system has no solutions.
            \item The system has exactly one solution.
            \item The system has infinitely many solutions.
        \end{enumerate}
        \item A linear system built from real data will almost never be directly solvable!
    \end{itemize}
\end{frame}

\begin{frame}{Working Towards a Solution}
    \begin{itemize}
        \item Let's rewrite our system of equations:
        \[
        r = 
        w_1 
        \left( 
        \begin{array}{c}
        v_{11} \\
        v_{21} \\
        \vdots \\
        v_{m1} 
        \end{array}
        \right)
        +
        w_2 
        \left( 
        \begin{array}{c}
        v_{12} \\
        v_{22} \\
        \vdots \\
        v_{m2} 
        \end{array}
        \right)
        + \ldots
        w_2 
        \left( 
        \begin{array}{c}
        v_{1n} \\
        v_{2n} \\
        \vdots \\
        v_{mn} 
        \end{array}
        \right)
        \]
        \item We can also ``tweak'' our objects!
        \begin{itemize}
            \item $w_i$ can be any mathematical object. For instance, a function.
            \item We can define our own concept of scalar multiplication\footnotemark[1].
            \item We can define our own concept of vector addition\footnotemark[1].
        \end{itemize}
        \footnotetext[1]{Some restrictions may apply!}
    \end{itemize}
\end{frame}

\begin{frame}{Enter Vector Spaces}
    \begin{itemize}
        \item A vector space consists of three objects:
        \begin{description}
            \item[$V$] A set. Elements $\vv{v} \in V$ are called vectors.
            \item[$\mathbb{F}$] A field. Elements $s \in \mathbb{F}$ are called scalars.
            \item[$\cdot$] Scalar multiplication operator: $s \cdot \vv{v}$.
            \item[$+$] Vector addition operation: $\vv{u} + \vv{v}$
        \end{description}
        \item A vector space must satisfy these axioms:
        \begin{itemize}
            \item Associativity of Addition] $(\vv{u} + \vv{v}) + \vv{w} = \vv{u} + (\vv{v} + \vv{w})$
            \item Zero  $\exists \vv{0}$ s.t. $\vv{u} + \vv{0} = \vv{u}$ 
            \item Negatives  $\forall \vv{u} \in V \exists -\vv{u}$ s.t. $\vv{u} + (-\vv{u}) = \vv{0}$
            \item Associativity of Multiplication $(ab)\vv{u} = a(b\vv{u})$ 
            \item Disributivity  $(a+b)\vv{u} = a\vv{u} + b\vv{u}$ and $a\vv{u} + b\vv{u} = (a+b)\vv{u}$
            \item Unitary  $1\vv{u} = \vv{u}$
        \end{itemize}
    \end{itemize}
\end{frame}

\begin{frame}{How will we use vector spaces?}
    \begin{itemize}
        \item We can often cast objects in such a way that the weight vector is solvable.
        \item Ultimately, we will explore tensors, which are multilinear maps:
        \[
        f : V_1 \times V_2 \times \ldots \times V_n \mapsto W 
        \]
        Where $V_i$ and $W$ are vector spaces.
    \end{itemize}
\end{frame}

\begin{frame}{Further Reading}
    I recommend reading over the following material before our next lecture:
    \begin{itemize}
        \item \href{https://machinelearningmastery.com/gentle-introduction-vectors-machine-learning/}{A Gentle Introduction to Vectors for Machine Learning}
        \item \href{http://www.math.toronto.edu/gscott/WhatVS.pdf}{What is a vector space?} by Geoffrey Scott
        \item \href{https://numpy.org/doc/stable/reference/generated/numpy.linalg.solve.html}{NumPy's Linear Algebra Solver}
    \end{itemize}
\end{frame}


\end{document}

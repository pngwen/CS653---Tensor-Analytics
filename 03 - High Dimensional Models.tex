% Uncomment for handout
%\def\HANDOUT{}


\ifdefined\HANDOUT
\documentclass[handout]{beamer}
\usepackage{pgfpages}
\pgfpagesuselayout{4 on 1}[letterpaper,landscape,border shrink=5mm]
\else
\documentclass{beamer}
\fi

\mode<presentation>
{
  \usetheme{Warsaw}
  \definecolor{sered}{rgb}{0.78, 0.06, 0.18}
  \definecolor{richblack}{rgb}{0.0, 0.0, 0.0}
  \setbeamercolor{structure}{fg=sered,bg=richblack}
  %\setbeamercovered{transparent}
}


\usepackage[english]{babel}
\usepackage[latin1]{inputenc}
\usepackage{times}
\usepackage[T1]{fontenc}
\usepackage{tikz}
\usepackage{graphicx}
\usepackage[export]{adjustbox}
\usepackage{fancyvrb}
\usepackage{amsmath}
\usepackage{amssymb}
\usepackage{esvect}

\newcommand{\imagesource}[1]{{\centering\hfill\break\hbox{\scriptsize Image Source:\thinspace{\tiny\itshape #1}}\par}}
\newcommand{\image}[2]{%
        \begin{center}
        \includegraphics[max height = 0.55\textheight, max width = \textwidth]{images/#1}
        \linebreak
        {\tiny Image Source:\thinspace{\tiny #2}}
        \end{center}
}

\newenvironment{code}{%
 \VerbatimEnvironment
 \begin{adjustbox}{max width=\textwidth, max height=0.7\textheight}
 \begin{BVerbatim}
  }{
  \end{BVerbatim}
 \end{adjustbox}
}

\title{03 - High Dimensional Models}


\author{Robert Lowe}

\institute[Southeast Missouri State University] % (optional, but mostly needed)
{
  Department of Computer Science\\
  Southeast Missouri State University
}

\date[]{}
\subject{}

\pgfdeclareimage[height=1.0cm]{university-logo}{images/semo-logo}
\logo{\pgfuseimage{university-logo}}



\AtBeginSection[]
{
  \begin{frame}<beamer>{Outline}
    \tableofcontents[currentsection]
  \end{frame}
}


\begin{document}

\begin{frame}
  \titlepage
\end{frame}

\begin{frame}{Outline}
  \tableofcontents
\end{frame}


% Structuring a talk is a difficult task and the following structure
% may not be suitable. Here are some rules that apply for this
% solution: 

% - Exactly two or three sections (other than the summary).
% - At *most* three subsections per section.
% - Talk about 30s to 2min per frame. So there should be between about
%   15 and 30 frames, all told.

% - A conference audience is likely to know very little of what you
%   are going to talk about. So *simplify*!
% - In a 20min talk, getting the main ideas across is hard
%   enough. Leave out details, even if it means being less precise than
%   you think necessary.
% - If you omit details that are vital to the proof/implementation,
%   just say so once. Everybody will be happy with that.

\section{A Little Review of Statistics}

\begin{frame}{Random Variable}
   \begin{itemize}
       \item A variable $X$ is random if it is the outcome of some random process.
       \item For the purpose of modeling, we treat all observed variables as random.
       \item We are typically interested in the distribution of values of a random variable, especially:
       \begin{itemize}
           \item Center of the distribution.
           \item Spread of the distribution.
       \end{itemize}
   \end{itemize} 
\end{frame}

\begin{frame}{Center}
    \begin{itemize}
        \item We typically use the arithmetic mean as our measure of the center of a distribution:
        \[
        \bar{X} = \dfrac{1}{n}\sum_{i=1}^n X_i
        \]
        \item In most distributions, values tend to be close to the arithmetic mean.
           So we often think of this as the expected value of $X$.
           \[
           E[X] = \bar{X}
           \]
    \end{itemize}
\end{frame}

\begin{frame}{Variance}
    \begin{itemize}
        \item Variance is the expected squared deviation of a random variable from its mean.
        \[
        \mathrm{Var}(X) = E[(X-\bar{X})^2]
        \]
        \item Assuming that the expected squared deviation is the mean squared deviation:
        \[
        \mathrm{Var}(X) = \dfrac{1}{n} \sum_{i=1}^n (X_i - \bar{X})^2
        \]
    \end{itemize}
\end{frame}


\begin{frame}{Standard Deviation}
    \begin{itemize}
        \item Standard Deviation is the unit correction of the variance.
        \[
        \sigma_x = \sqrt{\mathrm{Var}(X)}
        \]
        \item Or if we want to write a formula that looks good on a chalkboard:
        \[
        \sigma_x = \sqrt{\dfrac{1}{n} \sum_{i=1}^n (X_i - \bar{X})^2}
        \]
    \end{itemize}
\end{frame}

\begin{frame}{Normal Distribution}
    \begin{columns}
    \column{0.6\textwidth}
    \begin{itemize}
        \item The infamous bell curve!
        \item Specified $N(\bar{X}, \sigma)$
        \item Standard normal distribution: $N(0, 1)$
        \item Z-Scores are the projection into this distribution.
        \[
        z = \dfrac{X-\bar{X}}{\sigma_x}
        \]
    \end{itemize}
    
    \column{0.4\textwidth}
    \image{normal}{\href{https://en.wikipedia.org/wiki/Normal_distribution#/media/File:Normal_Distribution_PDF.svg}{Wikipedia}}
    \end{columns}
\end{frame}


\section{Multi-Variant Models}

\begin{frame}{Variance and Modelling}
    \begin{itemize}
        \item A model is a mathematical object which accounts for variance.
        \item The goal is to make random variables predictable.
        \item In multi-variant models, we want to identify causal relationships between variables.
    \end{itemize}
\end{frame}

\begin{frame}{Pairs of Variables}
    \begin{columns}
    \column{0.8\textwidth}
        \begin{itemize}
            \item Covariance measures the tendancy of two variables to change in similar ways.
            \[
            \mathrm{cov}(X, Y) = E[(X-E[X])(Y-E[Y])]
            \]
            \item Correlation is the measure of dependence between two random variables.
            \[
            \mathrm{corr} = \dfrac{\mathrm{cov}(X, Y)}{\sigma_x\sigma_y}
            \]
        \end{itemize}
    \column{0.2\textwidth}
        \image{covariance}{\href{https://en.wikipedia.org/wiki/Covariance#/media/File:Covariance_trends.svg}{Wikipedia}}
    \end{columns}
\end{frame}

\begin{frame}{Covariance Matrix}
    \[
    K=
    \left[ 
    \begin{array}{cccc}
        \mathrm{cov}(v_1, v_1) &  \mathrm{cov}(v_1, v_2) & \ldots & \mathrm{cov}(v_1, v_n) \\
        \mathrm{cov}(v_2, v_1) &  \mathrm{cov}(v_2, v_2) & \ldots & \mathrm{cov}(v_2, v_n) \\
        \vdots & \vdots & \ddots & \vdots \\
        \mathrm{cov}(v_n, v_1) &  \mathrm{cov}(v_n, v_2) & \ldots & \mathrm{cov}(v_n, v_n) \\
    \end{array}
    \right]
    \]
    \begin{itemize}
        \item Given a feature vector $\vv{v}$ of length $n$.
        \item Compute the the covariance of each pair, and organize them in a matrix:
        \[
        K_{i,j} = \mathrm{cov}(v_i, v_j)
        \]
    \end{itemize}
\end{frame}

\begin{frame}{The Curse of Dimensionality}
    \begin{itemize}
        \item Suppose we have four variables in our feature vector: $\vv{v} = \langle x, y, z, w \rangle$.
        \item We want to use these variables to predict some outcome $r$.
        \item We could have single variable relationships:
            $r = f(x)$, $r=f(y)$, $r=f(z)$, or $r=f(w)$
        \item We could have two variable relationships:
            $r = f(x, y)$, $r = f(x, z)$, $r = f(x, w)$, $r = f(y, z)$, $\ldots$
        \item We could have three variable relationships:
            $r = f(x, y, z)$, $r=f(x, y, w)$, $\ldots$
        \item We could have a four variable relationship:
            $r = f(x, y, z, w)$
        \item We could even have no relationship! 
    \end{itemize}
\end{frame}

\begin{frame}{Dimension Reduction}
    \begin{itemize}
        \item What if we could establish a new vector space for our observations?
        \item What if that space had fewer dimensions?
        \item Remember, the goal is to account for the variance in our observations!
        \item One method of dimension reduction is Principal Component Anlaysis (PCA).
    \end{itemize}
\end{frame}

\begin{frame}{Assignments}
    \begin{itemize}
        \item Read \href{https://setosa.io/ev/principal-component-analysis/}{Principal Component Analysis Explained Visually}
        \item Homework Assignment: \href{https://canvas.semo.edu/courses/8742/assignments/82308}{Principal Component Analysis with NumPy}
    \end{itemize}
\end{frame}

\end{document}
